
%%%%%%%%%%%%%%%%%%%%%%% file typeinst.tex %%%%%%%%%%%%%%%%%%%%%%%%%
%
% This is the LaTeX source for the instructions to authors using
% the LaTeX document class 'llncs.cls' for contributions to
% the Lecture Notes in Computer Sciences series.
% http://www.springer.com/lncs       Springer Heidelberg 2006/05/04
%
% It may be used as a template for your own input - copy it
% to a new file with a new name and use it as the basis
% for your article.
%
% NB: the document class 'llncs' has its own and detailed documentation, see
% ftp://ftp.springer.de/data/pubftp/pub/tex/latex/llncs/latex2e/llncsdoc.pdf
%
%%%%%%%%%%%%%%%%%%%%%%%%%%%%%%%%%%%%%%%%%%%%%%%%%%%%%%%%%%%%%%%%%%%


\documentclass[a4paper]{article}
\usepackage[utf8]{inputenc}
\usepackage{amssymb}
\usepackage{amsmath}
\usepackage{float}
\usepackage{color}


\newcommand\gilles[1]{\textcolor{red}{#1}}
\def\restriction#1#2{\mathchoice
              {\setbox1\hbox{${\displaystyle #1}_{\scriptstyle #2}$}
              \restrictionaux{#1}{#2}}
              {\setbox1\hbox{${\textstyle #1}_{\scriptstyle #2}$}
              \restrictionaux{#1}{#2}}
              {\setbox1\hbox{${\scriptstyle #1}_{\scriptscriptstyle #2}$}
              \restrictionaux{#1}{#2}}
              {\setbox1\hbox{${\scriptscriptstyle #1}_{\scriptscriptstyle #2}$}
              \restrictionaux{#1}{#2}}}
\def\restrictionaux#1#2{{#1\,\smash{\vrule height .8\ht1 depth .85\dp1}}_{\,#2}}

\begin{document}

% first the title is needed
\title{Energy aware scheduling}

\maketitle

\section{Problem formulation}

We consider a problem of applications scheduling in a center with limited power and resources.


This data center runs two different applications : active applications (also called web applications) and batch applications.

\subsection{Active applications}

We consider multiple active applications $\{A_{i}\}, i \in 1..n$, which run continuously over a given amount of time intervals $1..24$.

Each application $A_i$ has, at a given time interval $j$, an execution modes ; each application's mode produces a given power consumption and profit when applied over an interval, with
\begin{description}
 \item[$M_{i,j} \in 1..3$] mode of activity $i$ at time $j$.
 \item[$E_{i,j} \in 0..1000$] power consumption of activity $i$ at time $j$.
 \item[$P_{i,j} \in 0..1000$] profit for running activity $i$ at time $j$.
\end{description}

Those informations can be stored in tables, eg :

\begin{table}[H]
\begin{center}
  \begin{tabular}{ | l | l | l | l |}
     \hline
      & Mode 1 & Mode 2 & Mode 3 \\ \hline
    $A_1$ & 50 & 60 & 70 \\ \hline
    $A_2$ & 40 & 45 & 55 \\
    \hline
    \end{tabular}
\end{center}
\caption{Energy consumption for two active applications}
\end{table}

\begin{table}[H]
\begin{center}
    \begin{tabular}{ | l | l | l | l |}
    \hline
      & Mode 1 & Mode 2 & Mode 3 \\ \hline
    $A_1$ & 100 & 110 & 120 \\ \hline
    $A_2$ & 90 & 100 & 110 \\
    \hline
    \end{tabular}
\end{center}
\caption{Profit for two active applications}
\end{table}

\begin{itemize}
 \item element ($M_{i,j}$, $Row^{Energy}_{i}$, $E_{i,j}$) \% \quad $E_i$=$Row^{Energy}_{i} [M_i]$
 \item element ($M_{i,j}$, $Row^{Profit}_{i}$, $P_{i,j}$) \% \quad $P_i$=$Row^{Profit}_{i} [M_i]$
\end{itemize}

\subsection{Batch applications}

We consider multiple batch jobs $\{B_{i}\}, i \in1..m$, each with its own parameters :

\begin{description}
\item [$Duration_i$] is the number of intervals this job must be run before being finished,
\item [$Deadline_i$] deadline is the interval at which the job must be finished,
\item [$Q_i$] $\in 0..1000$ is the profit earnt only if the whole job is finished before or at deadline,
\item [$Power_i$] is the power consumed by the job when it is run over an interval
\end{description}
The batch jobs do not have execution modes.

\begin{table}[H]
\begin{center}
  \begin{tabular}{ | l | l | l | }
     \hline
      & On & Off \\ \hline
    $B_1$ & 60 & 0 \\ \hline
    $B_2$ & 80 & 0 \\
    \hline
    \end{tabular}
\end{center}
\caption{Energy consumption for Batch applications}
\end{table}

Each batch job $i$ can be decomposed in as many subjobs $i,j$ as its duration. We note $S_{i,j}$ the interval at which the $j^{th}$ subjob of jow $i$ is executed.
Since each subjob must be executed over a different time interval,\\
$0< S_{i,1}<S_{i,2}$...$<S_{i,Duration_i}$ .\\
If the job meets its deadline, then also\\
$S_{i,Duration_i}<=Deadline_i$

Energy consumption of batch job $i$ during an interval is $E_{i,j}$, where $i\in$ [1,m] and $j\in$ [1....$Duration_i$]. 
\begin{itemize}
 \item element ($S_{i,Duration_i}$,[.........$Profit_i$, $Pen_i,....$], $Q_i$)
\end{itemize}

\subsection{VM migration}
In the context of a data center made of more than one server, VMs are often migrated from one server to another. This migration is not costless in terms of energy consumption, we need to take those extra cost in our model. We introduce a migration cost both for active and batch jobs.

\begin{itemize}
\item[•] Active job.
Let $HA_{i,j}$ be the host of the application $A_i$ during interval $j$. We denote $P(HA_{i,j})$ the cost of moving the application $A_i$ from one host to $HA_{i,j}$. If $HA_{i,j-1}$ is defined i.e. application $A_i$ was running during interval $j-1$, and $HA_{i,j-1} = HA_{i,j}$ then $P(HA_{i,j}) = 0$. If $HA_{i,j-1}$ is undefined, then $P(HA_{i,j}) = 0$. If $HA_{i,j-1}$ is defined and $HA_{i,j-1} \neq HA_{i,j}$ then $P(HA_{i,j})$ is a function of $HA_{i,j-1}$ and $HA_{i,j}$.

\item[•] Batch job.
Let $HB_{i,j}$ be the host of the application $A_i$ during interval $j$. We denote $P(HB_{i,j})$ the cost of moving the application $B_i$ from one host to $HB_{i,j}$. If $HB_{i,j-1}$ is defined i.e. application $B_i$ was not yet completed during interval $j-1$, and $HB_{i,j-1} = HB_{i,j}$ then $P(HB_{i,j}) = 0$. If $HB_{i,j-1}$ is undefined, then $P(HB_{i,j}) = 0$. If $HB_{i,j-1}$ is defined and $HB_{i,j-1} \neq HB_{i,j}$ then $P(HB_{i,j})$ is a function of $HB_{i,j-1}$ and $HB_{i,j}$.

\end{itemize}

\subsection{Energy cap}

We also introduce the maximum available energy at $j$ as $Capacity_j$, where $j\in$ [1,24]. Since, in cumulative constraint modeling, the limit of maximum capacity of available energy can not be changed, we introduce fake jobs in each slot to match with the $Capacity_j$. So the maximum capacity is defined as Limit, where Limit = $max (Capacity_{h})$, ${h}\in$ [1,24]. For scheduling purpose we slice the total time i.e., 24/48 hours to 24/48 slots meaning 1 hour as slot and schedule each slot in advance with known information. So, Total Profit is $P=\sum_{i=1}^{n} {P_{i,j}} + \sum_{i=1}^{m} {Q_i}$
 
\begin{itemize}
\item For each active job $A_{i}$ starts at $j$, has duration of 1 slot and height of power consumption is $E_{ij}$, where, $\forall$j $\in$ [1,24]
\end{itemize}

\begin{itemize}
\item For each batch jobs $B_{i}$ the start time is $S_{ij}$, duration is 1 slot and and height of power consumption is $E_{i}$, where, $\forall$i $\in$ [1,m] and $\forall$j $\in$ [1,$duration_i$]
\end{itemize}

\begin{itemize}
\item  For every slot ${j}\in$ [1,24] we have a fake job $F_{h}$ that starts at fixed $j$, duration of 1 slot and height of power consumption of $Limit$ - $Capacity_j$, where, $\forall$j $\in$ [1,24]
\end{itemize}

\subsection{Memory use}

The data center contains $l$ servers $\{S_k\}, k\in 1..l$. Each server $S_k$ has a memory capacity $M_k$ which limits the number of applications this server can execute.

Executing an application $A_i$ or a job $B_i$ on a server $S_k$ consumes a fixed amount of memory 
on the server over the execution interval. Reciprocally, each application must be run on a server at any time and any job executed during an interval must be run on a server.

We note
\begin{description}
\item[$HA_{i,j}$] the host of the application $A_i$ during interval $j$
\item[$HB_{i,j}$] the host of the job $B_i$ during interval $j$
\item[$Load_{k,j}$] the memory load of the server $k$ during interval time $j$
\item[$mem(C)$] the memory use of the job or application C.
\end{description}

then we know that

\begin{equation}
Load_{k,j} = \sum_{A_i}(mem(A_i) if HA_{i,j}=k) + \sum_{B_i}(mem(B_i) if HB_{i,j=k}
\end{equation}

\section{Solution formulation}

A solution to such a problem consists in :
\begin{itemize}
\item $\forall$ Active application $A_i$ and time interval $j$, the execution mode $M_{i,j}$ at which to run the appliction during the interval
\item $\forall$ Batch job $B_i$, a series time interval $S_{i,1}..S_{i, Duration_i}$ at which to execute the job.
\item $\forall$ time interval $j$, $\forall$ server $S_k $, the set of Applications and jobs hosted on the server during the time interval.
\end{itemize}
with respect to the previoulsy specified conditions.

\section{Objective}
Our objective function is to maximize $P$,\\
$P= \sum_{j=1}^{24} \bigg( \sum_{i=1}^{n} {P_{i,j}}\bigg)  + \sum_{i=1}^{m} Q_i$

%$\gilles{+ \sum_{i=1}^{n} {P(HA_{i,j})} + \sum_{i=1}^{m} {P(HB_{i,j})} } $
\end{document}
